\chapter{Background}
    \label{chap:background}

   This chapter will cover the necessary technical background regarding Inter Component Communication. It will first give an overview of important concepts of the Android Operating System. Following that, we will explain what components and permissions are in Android, and then go on to discuss how components communicate between each other. Subsequently, we explore the various types of vulnerabilities related to Inter Component Communication. 
    
    \section{Basics of the Android Operating System}
        \label{sec:android_basics}
    
    In Android, each application is by default assigned a unique user ID known only by the OS. Each app runs in its own process by default, and each process runs in its own virtual machine \cite{android_app_fundamentals}. Consequently, apps are generally separated from each other, which enhances the security of the system.
    
    Throughout the continuous development of Android, the API is modified to introduce new features and improve security or performance. Therefore, in order to identify each incremental version of the API, a unique integer is assigned to each version or level. Over the years, there have been changes to the API that improved software security, and therefore some vulnerabilities are harder or impossible to exploit in current API levels.
    
    The manifest of an Android app is an XML file that gives the system information about the app’s structure, capabilities and needs. All Android app components, except broadcast receivers, need to be declared in the manifest file, and for each component you can define permission requirements and the capabilities of the component \cite{android_app_fundamentals}. Moreover, the developer can say in the manifest file what hardware or software system features the app uses. For example, an app would not be installed on a device if its manifest said it required a microphone and the mobile device did not posses microphone hardware.  Components will be explained in detail later in section \ref{sec:android_components}, and permissions in section \ref{sec:permissions}.
    
    \section{Android Components}
        \label{sec:android_components}
        
    Android mobile apps are made up of logical building blocks called components. A component is an entity which allows the user or the operating system to access the application \cite{android_app_fundamentals}. Therefore, a component does not necessarily correlate with other computing concepts such as processes or threads. When any component of an app needs to be run, the system starts a process for that app. There are four types of components in Android: activities, services, broadcast receivers and content providers. We will detail these in the rest of section \ref{sec:android_components}
    
    \subsection{Activities}
        \label{subsec:activities}
        
    Activities represent the individual app UI screens through which a user interacts with the app. For example, a news aggregator application might have an activity for viewing a list of news articles. Activities are used by the operating system to keep track of what the user sees on screen, what information they are interested in, and the information of minimized apps that might be needed later \cite{android_app_fundamentals}.
    
    \subsection{Services}
        \label{subsec:services}
        
    Services are components used for running long-term operations in the background. Importantly, a service does not represent a separate process or thread, but an interface for the system to let the app work in the background \cite{whats_is_a_service}. A service does not have a user interface itself. Examples of the usage of services include VPN apps that maintain a VPN connection in the background.
    
    There are three types of services: foreground services, which perform tasks that are noticeable to the user and must display a notification, background services, which do things that are not noticeable to the user, and bound services, which act as servers responding to requests made by client components \cite{services_overview}.
    
    \subsection{Broadcast Receivers}
        \label{subsec:receivers}
        
    Broadcast receivers are components used to receive system wide broadcasts. These broadcasts are messages sent by the operating system or by other apps. Applications can react to various events by using broadcast receivers. For example, the system can send a broadcast to let apps know that the device’s battery is low. An app can use a receiver to listen for an event even when the app is not running. Receivers do not have a user interface but can display notifications. In addition, it is worth noting that they do not have to be declared in the manifest but can be created programmatically as well.
    
    There are three types of broadcasts, two of which are relevant to our project:
    \begin{itemize}
        \item Normal broadcasts – These are sent to all receivers at the same time, and each receiver can react independently of other receivers.
        \item Ordered broadcasts – These are sent to receivers one at a time. Unlike with a normal broadcast, the receiver currently processing the broadcast can change what information the broadcast contains, and can even cancel the broadcast, so that it will not be sent to further receivers \cite{broadcasts_overview}. Broadcast receivers can be registered with a certain priority for getting broadcasts.
    \end{itemize}
    
    \subsection{Content Providers}
        \label{subsec:content_providers}
        
    Content providers are interfaces through which apps can access data stored in persistent storage such as a remote server, an SQL database or local file storage. A provider can be used by components of the same app or by components of other apps. Therefore, they are used by the system to manage access to shared data. Content providers can restrict access to the data to apps with certain permissions and give temporary access to certain files only \cite{android_app_fundamentals}.
    
    \section{Permissions}
        \label{sec:permissions}
        
    Android follows the principle of least privilege. This principle is enforced through a system of permissions, meaning that an application can only access sensitive data, system features or components of other applications if it possesses the necessary requirements \cite{permissions_guide}. For instance, an application needs the correct permission to access the user’s contacts. 
    
    Moreover, developers can protect a component of an app with permission requirements by adding an \lstinline|android:permission| tag in the manifest file. Only components in apps that have that permission will be able to send an intent to the protected component.
    
    \subsection{Types of permissions}
        \label{subsec:types_of_permissions}
        
    There are four types of permissions, three of which are relevant to this project:
    
    \begin{itemize}
        \item Normal permissions – Permissions for unimportant resources, such as the permission to set the time zone \cite{permissions_guide}. They are granted automatically at install time.
        \item Dangerous permissions – They are for important resources such as private user information, or that can affect the state of the system or of other apps. The user needs to give explicit permission in the app to utilise these resources.
        \item Signature permissions – These are special permissions designed for use among a group of apps created by the same developer. An app is automatically granted a signature permission at install time only if it is signed by the same certificate as the app that defined the permission. The certificate can be self-signed by the developer. Its purpose is to identify the author of an app \cite{define_custom_permission}.
    \end{itemize}
    
    \subsection{Custom permissions}
        \label{subsec:custom_permissions}
        
    Applications can declare their own permissions. These can be used to restrict access to components of an application, or protect broadcasts of that app. This is done by declaring a permission in the manifest file of the app, as you can see in listing \ref{lst:custom_permission}.
    
    \lstinputlisting[language={xml}, label={lst:custom_permission}, caption={A declaration of a custom permission in an Android manifest.}]{./listings/custom_permission.xml}
    
    \section{Inter Component Communication}
        \label{sec:inter_component_communication}
    
    So far, we have seen that each Android application runs in its own sandbox, and by default can not see what other applications are doing. Sometimes, we need the system to communicate with the apps, and applications can enrich the users experience by collaborating. Moreover, an application component can be used by other apps to provide extra functionality. For example, a browser lets you select which social media or messaging app to use for sharing a link.
    
    Intents are a class in the Android API that are used as messages for communication between application components. More specifically, intents are used to start new activities, start and stop services, bind or unbind a component to a service, and they also represent the broadcasts that are sent to receivers. Intents can carry data in the form of a URI, as well as other data in the form of key value pairs \cite{intents}.
    
    \subsection{Exported Components}
        \label{subsec:exported_components}
        
    By default, app components are not accessible to outside apps through intents. However, a component can be exported and thus receive intents from other applications. To export a component, you can set the \lstinline|<exported>| tag in a component in the app’s manifest to true. However, if the component has an intent filter defined in the manifest, the component will become automatically exported unless the exported tag is explicitly set to false. Intent Filters will be fully explained in section 3.5.3.
    
    Further complicating component exportation is that developers can configure an application to use the same Android User ID as other applications created by them. This means these apps can run in the same process, and they can access each other’s components regardless of the exported tag or the presence of intent filters. This is shown very well in Table 1 in \cite{2013_permission_leaks_study}.
    
    \subsection{Explicit Intents}
        \label{subsec:explicit_intents}
        
    Explicit intents directly specify the application that should receive the intent and handle it. This is done by setting either the package name of the receiving application, or the full name of a component of said app \cite{intents_and_intent_filters}. Explicit intents can contain other information, such as data or the intended action to be performed, as you can see in Listing \ref{lst:explicit_intent}.
    
    Using an explicit intent means that only the targeted app or component can receive the intent. Explicit intents are usually used for communication between components of the same app, such as when one activity starts another when the user clicks a button. That being said, explicit intents can be used to start components of other apps as well. As explained in subsection \ref{subsec:exported_components}, an app component must be exported so that other apps can send explicit intents to it.
    
    \lstinputlisting[language={java}, label={lst:explicit_intent}, caption={Kotlin code to make an explicit intent, add data to it and start an activity with it}]{./listings/explicit_intent.kt}
    
    \subsection{Implicit Intents}
        \label{subsec:implicit_intents}
        
    Unlike explicit intents, implicit intents do not directly specify what application or component it should be sent to. Instead, the Android system decides who to send it to based on the information in the intent and what other components have declared they can handle.
    
    A component defines what intents it can handle by specifying Intent Filters in the manifest file, with an example in Listing \ref{lst:intent_filter}. An Intent Filter defines the type of intents an application can handle. A filter can say what actions the component can perform, what intent categories it accepts, the MIME data types it accepts or the kind of URI resources it can handle. A component may declare multiple Intent Filters, and it is recommended that this is done for each task the component can do \cite{intents_and_intent_filters}.
    
    \lstinputlisting[language={xml}, label={lst:intent_filter}, caption={Declaration of an intent filter that the intent in Listing \ref{lst:implicit_intent} will match.}]{./listings/intent_filter.xml}
    
    When an implicit intent is sent, like you can see in Listing \ref{lst:implicit_intent}, the Android System compares its attributes against all intent filters of all components. For the intent to be matched with a filter, three tests are performed: the Action test, the Category test, and the Data test \cite{intents_and_intent_filters}. In order to pass the Action test, the Intent’s action must be amongst the actions of the filter. It passes the Category Test if all of its categories are found in the filter’s declaration, and the Data Test is passed if the data URI or MIME type of the intent matches one of the data elements in the filter. If the component has multiple filters, the intent only needs to match one of them for it to be passed to the component.
    
    \lstinputlisting[language={java}, label={lst:implicit_intent}, caption={Kotlin code to make an implicit intent, add data to it and start an activity with it}]{./listings/implicit_intent.kt}
    
    If only one intent filter matches the implicit intent, the operating system will start that filter’s component automatically. However, if there are multiple matches, a dialog will be displayed to the user so they can manually select the component to handle the intent.
    
    For example, if there are multiple browsers installed on a device, and within an app the user clicks on a web link, they will then see an Android dialog letting them choose what browser to use to open that page. This is because the parent app sent an implicit intent, and all browsers had filters that matched with the intent.
    
    \section{Inter Component Communication Vulnerabilities and Attacks}
        \label{sec:icc_vulnerabilities_and_attacks}
        
    In this section, we will explain how the way components communicate using Intents can be exploited by attackers, and what developers can do to fix these vulnerabilities.
    
    Most of the vulnerabilities that will be explored in this project happen due to the misuse of implicit intents or intent filters. Because implicit intents do not directly state what component they target, it is possible that an intent is delivered to a malicious app. Moreover, an attacker could create malicious intents that could launch other components in ways that could compromise the victim app.
    
    \subsection{Intent Hijack}
        \label{subsec:intent_hijack}
        
    The Android documentation recommends that explicit intents are used for intra-app communication, and implicit intents for inter-application communication \cite{intents_and_intent_filters}. However, developers sometimes use implicit intents to start a component within the same app. An attacker can make an app with an intent filter designed to match with said implicit intents, which can direct the intent to the malicious application. The process in which an intent is matched against a filter was described in section \ref{subsec:implicit_intents}. When receiving an intent, a component can read all of its data. Therefore, even if the implicit intent is meant for external use, if the developer puts sensitive information in it, that data could be intercepted.
    
    In general, vulnerabilities against this class of attacks are fixed by using explicit intents instead of implicit intents in order to send broadcasts, start activities or services or grant access to a content provider \cite{2010_icc_paper}. Another way to mitigate these attacks is to not put sensitive information in implicit intents.
    
    \subsection{Intent Spoofing}
        \label{subsec:intent_spoofing}
        
    While Intent Hijack attacks work by accidentally activating a malicious component when an intent is intercepted, Intent Spoofing attacks happen when a victim component is unexpectedly activated by an attacking component using an Intent. Often, this attack targets components that are not meant to be accessible outside of their apps, but because they have an intent filter and the \lstinline|<exported>| tag is not set, they are exported automatically. Because exported components are accessible to other apps, the attacker can create an explicit intent targeting it and does not need to worry about having to match intent filters. The developers are usually unaware of this.
    
    This attack is dangerous because components often extract information from the intents they receive. An Intent Spoofing attack could insert malicious information into the victim component. The attacker could send invalid information to crash the victim’s app (a DoS attack), corrupt the user’s data, or inject commands to retrieve data. Often, the launch of an activity, service or broadcast receiver leads a change in the application’s state \cite{2010_icc_paper}, even if the victim component takes no input from the intent. And since activities and services can return information to the component that activated them, they could leak sensitive information. This attack is often done against components that were not meant to be accessible to other apps, and these components are thus less likely to check the input data of the intent or make sure they do not return private data.
    
    Developers can secure their application from this class of attacks in a number of ways  \cite{2010_icc_paper}. First, they should not declare intent filters for internal components, and they should use explicit intents to launch them. Secondly, if they need to use intent filters for internal components, they should explicitly declare them as not exported in the manifest. Thirdly, they can protect components with permissions of various types. Fourthly, state-changing operations should not be done in exported components. And finally, developers should make sure that exported components do not return information that should be private.
    