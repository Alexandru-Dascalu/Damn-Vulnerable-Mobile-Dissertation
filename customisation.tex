%% Here you can specify new commands and environments that you intend
%% to use. Using commands can make your document easier to write, read
%% and be more consistent.

\usepackage[linesnumbered,ruled]{algorithm2e}
\DeclareMathOperator*{\argmin}{arg\,min}

\usepackage{appendix}
\usepackage{textcomp}
\usepackage{setspace}
%\usepackage[document]{ragged2e}
\usepackage{verbatim}
\usepackage{multirow}
\usepackage{multicol}
\usepackage{booktabs}
\usepackage{enumitem}
\sloppy
\usepackage{graphicx}
\usepackage{threeparttable}
\usepackage{epsfig}
\usepackage{epstopdf}
\usepackage{float}
\usepackage{enumitem}
\usepackage{cite}
\usepackage[export]{adjustbox}
\usepackage{algorithmic}
\usepackage[nohyperlinks,printonlyused]{acronym}
\usepackage{amsmath}
\usepackage{amsfonts}
\usepackage{array}
\usepackage{tabularx}
\usepackage{longtable}
\usepackage{times}
\usepackage{amssymb}
\usepackage{hhline}
\usepackage{color}
\usepackage{soul}
\usepackage{colortbl}
\definecolor{Gray}{gray}{0.85}
\usepackage{rotating}
\usepackage{fix2col}
\usepackage{pdflscape}
\usepackage{pdfpages}
\usepackage{stmaryrd}
\usepackage[export]{adjustbox}
\usepackage{bbm}
\usepackage{relsize}
\usepackage{xfrac}
\usepackage{bibentry}
\usepackage{wrapfig}

%\usepackage{refcheck}

%watermarking
\usepackage[english]{babel}
\usepackage{tikz}

%% Uncomment the following line for hyper links - not recommended for printing
\usepackage[colorlinks, linkcolor=, anchorcolor=, citecolor=, filecolor=, menucolor=, runcolor=, urlcolor=]{hyperref}

\setcounter{tocdepth}{1}
%\setcounter{minitocdepth}{3} 
\linespread{1.31}

\newcommand\litem[1]{\item{\bfseries #1:\enspace}}

\interdisplaylinepenalty=2500

\newcolumntype{L}[1]{>{\raggedright\let\newline\\\arraybackslash\hspace{0pt}}m{#1}}
\newcolumntype{C}[1]{>{\centering\let\newline\\\arraybackslash\hspace{0pt}}m{#1}}
\newcolumntype{R}[1]{>{\raggedleft\let\newline\\\arraybackslash\hspace{0pt}}m{#1}}

\renewcommand{\thefootnote}{\fnsymbol{footnote}}
\setlength{\LTpre}{-10pt}\setlength{\LTpost}{-30pt}%
\newcommand{\oiint}{\begin{picture}(0,0)(-10,-2)\put(0,0){\oval(12,8)}\end{picture}\iint}
\renewcommand{\mathbf }{\boldsymbol}

\def \eg{e.g.\ } % Allows you to write \eg in LaTeX instead of typing e.g. so that every single one will be formatted the same.
\def \Eg{E.g.\ } % Define some other common variants. If you later want to change one of these definitions, 
\def \ie{i.e.\ } % it will update all the usages throughout the document.
\def \Dr{Dr.\ }
\def \vs{vs. }
\def \etal{\emph{et al.\ }} 
\def \sota{state-of-the-art }
\def \handcrafted{hand-crafted }

\usepackage{listings,lstautogobble}
\usepackage{sourcecodepro}
\pdfmapfile{=SourceCodePro.map}
\lstset{
	xleftmargin=0.5cm,frame=tlbr,framesep=4pt,framerule=0.5pt,
	language=,
	upquote=true,
	columns=fixed,
	tabsize=2,
	extendedchars=true,
	breaklines=true,
	numbers=left,
	numbersep=10pt,
	basicstyle=\ttfamily\scriptsize,
	numberstyle=\tiny,
	stringstyle=\ttfamily,
	captionpos=b,
	showstringspaces=false,
	autogobble=true
}

\usepackage[font=small,skip=10pt]{caption} %,format=hang
%\usepackage[labelformat=simple]{subcaption}
\usepackage[labelformat=simple]{subfig}
%\captionsetup[figure]{format=hang}
%\captionsetup[lstlisting]{format=hang}
\renewcommand{\thesubfigure}{\Alph{subfigure}.}

\renewcommand{\thefootnote}{\arabic{footnote}}

% Use IEEEtran citation style. 
\bibliographystyle{IEEEtran} 

\def\samplefont#1{%
    % set font style and save name
    #1\edef\savedname{\fontname\font}%
    % print small sample
    {\leavevmode\tt\hbox to 1in{\savedname:\hss}}%
    abcxyz ABCXYZ 123\par
}