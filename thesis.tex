% Two sided means the left and right margins are different sizes and they alternate every page. 
% If your document is printed to be book or spiral bound this allows for a thick spine to not 
% eat into the space for your page content. 
\documentclass[11pt, a4paper, twoside, openany]{custard}

\usepackage[normalem]{ulem}

% All imports, packages, and configuration in here. 
% Your document should be about content so we abstract away the styling rules and tools we are using. 
\input{customisation}
\begin{document}
	
% The custom data for Swansea University and your degree name.
% The \protect\\ command forces a new line in the title which might be otherwise overriden by the template
	\title{Intentionally insecure Android Image against Inter Component Communication attacks}
	\author{Alexandru Dascălu\protect\\{\normalsize 965337}}
	\awardinginst{Swansea University}
	
% Comment / uncomment your degree type as needed.
	\degree{Bachelor of Science} 
	
% Institution details and logo
	\department{Department of Computer Science}
	\university{Swansea University}
	\unilogo{graphics/swansea.png}
	
% Hard code the date or allow the LaTeX compiler to fill it in whenever you recompile the document.
	\date{\today}
	
% Build the title and declaration pages, and pad the document so the text starts on a right hand book page.
% Page numbering is in roman numerals until the first page of an actual chapter which resets numbers 
% starting from 1 at that point. 
	\frontmatter%
	\maketitle
	\declaration
	\cleardoublepage
	
% Most books and theses have a brief foreword or dedication. 
	\begin{vplace}[0.7]
		\begin{large}
			\begin{center}
				\textit{I would like to dedicate this work to my family, for raising me and supporting me to be where I am now.}
			\end{center}
		\end{large}
	\end{vplace}

% Abstract comes before the contents page.	
	\begin{abstract}
		\vspace{-2em}
		\setcounter{page}{1}
		
		In the modern world, mobile devices have surpassed laptops and desktops in importance for many people. A negative effect of this is that mobile apps have become attractive targets for attackers. Cyber security is weak in many applications. Therefore, some have developed intentionally insecure apps which teach developers about vulnerabilities so that they can make their software more secure. One class of vulnerabilities that is both dangerous and is not focused on by these projects is related to Android Inter Component Communication.
		
		Components are the building blocks of Android apps, and they can communicate with each other through objects of the Intent class. Due to the current implementation of Inter Component Communication in Android, malware can control or spy on a vulnerable app. Many of the existing Android intentionally insecure apps only partially focus on this type of vulnerability, do not offer real and authentic examples of malicious and vulnerable apps, and may even require the user to use command-line tools to perform an attack.
		
		We have developed the DVM-ICC app, which is an educational tool for exploring in detail various types of ICC vulnerabilities. It contains educational content, provides interactivity, and allows the user to perform and observe cyber attacks in the form of challenges. We also created a suite of apps, with each acting either as a vulnerable or as a malicious app in a challenge. The user can see an example of malware attacking a vulnerable app and can complete the challenges on their mobile device.
		
		The insecure suite of Android apps we created fulfills its aim of educating users about ICC vulnerabilities and raising awareness, as demonstrated by two structured interviews I conducted. It is a polished piece of work that provides real examples of vulnerable and malicious apps, is very interactive, focuses exclusively on ICC vulnerabilities and attacks, provides plenty of guidance, and can be used on a device, without a computer. Thus, it improves in key areas that are lacking in other projects of this type.
	\end{abstract}
	
% A long form dedication. 
	\begin{Acknowledgements}
		First of all, I would like to express my deep gratitude to my mother for all the emotional support and encouragement that she has given me over the past academic year and for being there when I needed her.
		
		Furthermore, I would like to thank Dr Phillip James for being a great supervisor, being easy to contact and guiding me towards realistic project goals.
		
		I am thankful towards my colleagues Constantinos Loizou and Avi Varma for their support and suggestions, and want to thank Avi for sharing with me things he learnt when doing his project a year ago. Moreover, I want to thank them for participating in an informal user study to help me evaluate my project.
		
		I sincerely thank Phillippos Pantekis and Bessam Helal for advising me during the process of picking my project in May 2020.
		
		Finally, I would like to thank Dr Mihaela Cojocaru for helping me to improve my mental health during this difficult time.
	\end{Acknowledgements}
	
% Build the table of contents page.
	\tableofcontents*
	 
% Optionally you can make a bank of known acronyms in acronyms.tex that you can call on throughout your document.
	%\input{acronyms} 
	
% For long documents like a Doctoral thesis you should include a list of tables and figures throughout 
% your document. This is uses a shortened version of each table and figures caption and enumerates all 
% of them with their table or figure number. This is automatic, you do not need to modfy it if you do use it.
	\input{list_of_tables_and_figures} 
	
% Reset numeric page numbering from page 1
	\mainmatter%

% Insert the code for each of your chapters
	\chapter{Introduction}
	\label{chap:intro}
	
	\section{Current State of Cyber Security}
		\label{sec:intro_motivation} 
		
		Ever since the release of the iPhone in 2007, smart phones and other mobile devices have played an increasingly larger role in our lives. We are using mobile devices for education, social interaction, commerce, mobile banking, entertainment, work, and more. 
		
		In the first world, mobile device usage has become ubiquitous. In the UK, smartphones became the most widely owned internet-enabled device in 2015 \cite{ofcom_coms_report}. Moreover, 52.6 percent of global web traffic came from mobile devices in 2019, up from 31.6 per cent in 2015 \cite{statista_mobile_web_traffic}. 
		
		Given the statistics from above, you would hope that mobile app developers have learned lessons from their desktop counterparts, and therefore, the majority of mobile apps are reasonably secure.
		
		That assertion proves to be false. A study done by cyber security consultancy Positive Technologies analysed 17 mobile apps, 8 for Android and 9 for iOS, through comprehensive penetration testing in 2018 \cite{pt_mobile_apps_2019}. On the client side, it found that only 11\% of the apps had an acceptable level of security, with 56\% having a low or below average level of security, and 43 percent of Android apps had critical vulnerabilities. On the server side, security was no better, with 57 percent of components having low or below average security, and a third of components had critical vulnerabilities. A more worrying piece of information from the report is that the average Android client app in the study contained 3.7 vulnerabilities, with 1.1 of those being of high risk.
		
		Against the lacklustre security of mobile apps, attackers are becoming more relentless and cunning, and are using more elaborate techniques \cite{pt_threatscape_2018}. In figure \ref{fig:no_attacks_evolution} you can see
		how the monthly number of reported cyber attacks has tended to increase since 2017. This situation has been exacerbated by the COVID-19 pandemic \cite{fbi_covid_fraud}, \cite{guardian_covid_attack}, with coronavirus related cyberattacks targeting both mobile device users \cite{canada_covid_attack} and hospitals \cite{czech_hospital_covid_attack}.
		
		\begin{figure}[h]
            \centering
            \includegraphics[width=1\textwidth]{graphics/threatscape_chart.PNG}
            \caption{Increasing number of monthly cyber attacks since 2017. Data is compiled from \cite{pt_threatscape_2018}, \cite{pt_mobile_apps_2019}, \cite{pt_threatscape_2020} and              \cite{pt_threatscape_2020q2}.}
            \label{fig:no_attacks_evolution}
        \end{figure}
		
		One of the categories of vulnerabilities that the authors of \cite{pt_mobile_apps_2019} analysed was Inter Process Communication related vulnerabilities. Inter Process Communication, abbreviated as IPC, refers to any exchange of information between two or more running processes on a computer system. In regards to Android, \cite{pt_mobile_apps_2019} also uses the term Inter Component Communication, or ICC, interchangeably with IPC. Android components are the logical building blocks of an app, and components can communicate with components, which can belong to the same app or to another app on the same device. Android components, especially if they are of the same app, can run in the same process. Therefore, IPC and ICC are not synonymous, even though they often overlap.
		
		In the study conducted in \cite{pt_mobile_apps_2019}, 29 percent of of client apps used insecure inter-process communication.
	
	\subsection{Objective}
		\label{sec:intro_objective} 
		
		In this document we present a tutorial on thesis creation and typesetting, and discuss topics such as literature surveying and proper citation. 
		
	\section{Overview}  
		\label{sec:intro_overview} 
		
		The remainder of chapter \ref{chap:intro} outlines the document structure and the key contributions of this work is organized as follows. Chapter \ref{chap:resources} reviews techniques for finding and properly citing external resources from the academic literature and online. In chapter \ref{chap:typesetting} we show examples of how to typeset different types of content, such as internal references, figures, code listings, and tables. And lastly in chapter \ref{chap:conclusion} we summarize the main contributions and key points to take away from this template.
	
	\section{Contributions} 
		\label{sec:intro_contribs} 
		
		The main contributions of this work can be seen as follows:
		
		\begin{description}	
		
			\item[$\bullet$ A LaTeX thesis template]\hfill
			
			Modify this document by adding additional TeX files for your top level content chapters. 
			
			\item[$\bullet$ A typesetting guide of useful primitive elements]\hfill
			
			Use the building blocks within this template to typeset each part of your document. Aim to use simple and reusable elements to keep your LaTeX code neat and to make your document consistently styled throughout.
			
			\item[$\bullet$ A review of how to find and cite external resources]\hfill
						
			We review techniques and resources for finding and properly citing resources from the prior academic literature and from online resources.
			
		\end{description}
	\chapter{Background}
    \label{chap:conclusion}

   This chapter will cover the necessary technical background regarding Inter Component Communication. It will first explain Inter Process Communication (IPC) in computing in general. Next, it will give an overview of important concepts of the Android Operating System. Following that, we will explain what components and permissions are in Android, and then go on to discuss in detail Inter Component Communication, which is a type of IPC specific to this OS. Subsequently, we explore the various types of vulnerabilities related to Inter Component Communication. 

    \section{Inter Process Communication} 
        \label{sec:ipc}

    In the context of computing, a process refers to a running instance of a computer program. A process has its own section of computer memory that it has access to. By default, processes cannot access the memory address space of another process. Processes have one or more threads, which are individual sequences of instructions executed by the computer at a given time. 
    
    Inter Process Communication refers to the ways in which processes communicate and share data between each other. Using various means of IPC, computer programs can enhance their functionality greatly. For example, a word processing software like Microsoft Office Word can get data from a spreadsheet through Excel.
    
    IPC can be accomplished through a variety of means: sharing files, sharing memory via various APIs, pipes, network sockets, the clipboard, and various means specific to each Operating System.
    
    \section{Basics of the Android Operating System}
        \label{sec:android_basics}
        
    Android is an open source operating system designed for use on mobile devices released in September of 2008 \cite{android_release_date}. It is built around a customised version of the Linux kernel. The UI and API are written in Java \cite{android_languages_statistic}, which was also the primary development language for apps until 2019 when it was replaced by Kotlin \cite{android_using_kotlin}. Developer tools package compiled code along with any necessary resource files into an archive file called an APK, or Android package, which is then used by a mobile device to install the application.
    
    In Android, each application is by default assigned a unique user ID known only by the OS. The files of an app can only be accessed by a Linux user with the same ID as that of the app. Moreover, each app runs in its own process by default, and each process runs in its own virtual machine \cite{android_os_fundamentals}. Consequently, each Android application runs in its own sandbox, which enhances the security of the system, as apps are generally separated from each other.
    
    Throughout the continuous development of Android, the API is modified to introduce new features and improve security or performance. Therefore, in order to identify each incremental version of the API, a unique integer is assigned to each version or level. Over the years, there have been changes to the API that improved software security, and therefore some vulnerabilities are harder or impossible to exploit in current API levels.
    
    The manifest of an Android app is an XML file that gives the system information about the app’s structure, capabilities and needs. All Android app components, except broadcast receivers, need to be declared in the manifest file, and for each component you can define permission requirements and the capabilities of the component \cite{android_app_fundamentals}. Moreover, the developer can say in the manifest file what hardware or software system features the app uses, whether those features are required, and what is the minimum API level the app requires. For example, an app would not be installed on a device if the app’s manifest said it required a microphone and the mobile device did not posses microphone hardware.
    
    The manifest also specifies the target SDK version of the app, which refers to the API version for which the is built to run on. Android apps can run on devices with a different API version as well.
    
    Components will be explained in detail later in section 3.3, and permissions in section 3.4.
    
    \section{Android Components}
        \label{sec:android_components}
        
    Android mobile apps are made up of logical building blocks called components. A component is an entity which allows the user or the operating system to access the application \cite{android_app_fundamentals}. Therefore, a component does not necessarily correlate with other computing concepts such as processes or threads. When any component of an app needs to be run, the system starts a process for that app.
    
    There are four types of components in Android: activities, services, broadcast receivers and content providers. We will detail these in the rest of section \ref{sec:android_components}
    
    \subsection{Activities}
        \label{subsec:activities}
        
    Activities represent the individual app UI screens through which a user interacts with the app. For example, a news aggregator application might have an activity for viewing a list of news articles.
    
    Activities are used by the operating system to keep track of what the user sees on screen, what information they are interested in, and the information of minimized apps that might be needed later \cite{android_app_fundamentals}.
    
    Activities are independent components \cite{android_app_fundamentals}, and therefore it is possible that activities of different apps collaborate. This is what happens in a browser when the user wants to share a link and they choose between various messaging applications.
    
    \subsection{Services}
        \label{subsec:services}
        
    Services are components used for running long-term operations in the background. Importantly, a service does not represent a separate process or thread, but an interface for the system to let the app work in the background \cite{whats_is_a_service}. A service does not have a user interface itself. Examples of the usage of services include VPN apps that maintain a VPN connection in the background.
    
    There are three types of services: foreground services, which perform tasks that are noticeable to the user and must display a notification, background services, which do things that are not noticeable to the user, and bound services, which act as servers responding to requests made by client components \cite{services_overview}.
    
    \subsection{Broadcast Receivers}
        \label{subsec:receivers}
        
    Broadcast receivers are components that an app uses to receive system wide broadcasts. These broadcasts are messages sent by the operating system or by other apps. Applications can react to various events by using broadcast receivers. For example, the system can send a broadcast to let apps know that the device’s battery is low or that airplane mode has been activated. An app can use a broadcast receiver to listen for an event even when the app is not running. Broadcast receivers do not have a user interface but can display notifications in the device’s status bar. In addition, it is worth noting that they do not have to be declared in the manifest but can be created programmatically as well.
    
    There are three types of broadcasts:
    \begin{itemize}
        \item Normal broadcasts – These are sent to all receivers at the same time, and each receiver can react independently of other receivers.
        \item Ordered broadcasts – These are sent to receivers one at a time. Unlike with a normal broadcast, the receiver currently processing the broadcast can change what information the broadcast contains, and can even cancel the broadcast, so that it will not be sent to further receivers \cite{broadcasts_overview}. Broadcast receivers can be registered with a certain priority for getting broadcasts.
        \item Sticky broadcasts – They are persistent broadcasts, they remain after they have been broadcast to all receivers and are re-broadcast to any new receivers. They have been deprecated since API level 21, because they are very insecure \cite{sticky_broadcast}.
    \end{itemize}
    
    \subsection{Content Providers}
        \label{subsec:content_providers}
        
    Content providers are interfaces through which apps can access data stored in persistent storage such as a remote server, an SQL database or local file storage. A provider can be used by components of the same app or by components of other apps. Therefore, they are used by the system to manage access to shared data. Content providers can restrict access to the data to apps with certain permissions and give temporary access to certain files only \cite{android_app_fundamentals}.
	\chapter{Implementation}
    \label{chap:implementation}

   This chapter will explain my implementation for this project. It will first give an overview of the architecture of the software product. After that, we will present each one of the challenges that have been implemented so far.

    \section{Architecture} 
        \label{sec:architecture}
        
    The central part of the software implementation consists of the Damn Vulnerable Mobile - Inter Component Communication app. This Android app contains all of the educational material needed to understand Android ICC and complete each challenge, encourages the user to learn interactively,  and it provides access to each challenge and to settings to control the challenge or the learning experience. 
    
    The rest of the project consists of a series of Android apps that play the role of either a vulnerable app or a malware. These apps are made to appear to be authentic apps with real-world functionalities, though many of them are nowhere close to full-feature app. Each challenge consists of an authentic scenario of a malware app attacking one of the vulnerable apps through one particular ICC vulnerability.
    
    The DVM-ICC app communicates with the challenge apps through the use of files in order to apply various settings to control the behaviour of the apps. These settings are explained in subsection \ref{subsec:challenge_settings}. Moreover, each pair of vulnerable and malicious apps communicate between each other as a result of the cyber attack, through the medium of Intents. The manner in which this communication takes place will be explained in detail for each challenge in section \ref{sec:challenges}.
    
    \section{DVM-ICC Application}
        \label{sec:home_app}
        
    In this section, I will explore the features of the DVM-ICC Android app. While doing so, I will explain the intended workflow that a user needs to follow and how app provides an educational experience. The follow description applies to the Beginner mode. The app has three different operation modes, which will be fully explained in subsection \ref{subsec:challenge_modes}.
    
    \subsection{Home Menu}
        \label{subsec:home_menu}
        
    \begin{wrapfigure}[15]{r}{0.3\textwidth}
        \centering
        \includegraphics[width=0.3\textwidth]{graphics/home_activity.PNG}
        \caption{The first menu in the DVM-ICC app.}
        \label{fig:home_menu}
    \end{wrapfigure}
        
    The first screen the user sees when starting the app has 3 sub-pages: Introduction, Info and Challenges, as you can see in figure \ref{fig:home_menu}. The Introduction page gives an introduction of the project and explains the intended workflow for using the product. The Info page provides all of the necessary technical background to understand ICC vulnerabilities and attacks. It covers what components and permissions are, how components communicate between each other and the major types of ICC based attacks: Intent Hijacking and Intent Spoofing. The Challenges page lets the user select what challenge they want to start, as shown in figure \ref{fig:home_menu}.
    
    \subsection{Challenge Settings}
        \label{subsec:challenge_settings}
        
    When clicking on a challenge, the user is taken to a menu where they can change the settings that define how the challenge will be undertaken. Here, you can set the operation mode for the challenge. Moreover, the user can enable or disable the malware of the selected challenge. If disabled, the malicious app will not perform any cyber-attack and therefore it will not interfere with any vulnerable app or the rest of the system.
    
    The most important setting on this screen is the security level of the vulnerable app. Inspired from DVWA, as described in subsection \ref{subsec:ICC_related_work}, these levels define how secure the vulnerable app is against attacks from the malware. Each challenge's vulnerable app has between 2 and 5 security levels, with each successive level using more secure code. This culminates with the Impossible level, where the vulnerability is fixed and the malware can not perform the attack. When the malware is enabled, it will overcome the defences of the current security level, except for the Impossible level. The number of security levels and their meaning depends on the challenge. In general, each security level means that a component or broadcast is protected with a more powerful permission, that components are no longer exported or that an intent is now sent explicitly. 
    
    These settings are written to a text file that is located in the DVM-ICC's app specific directory on the device's external storage, from where it can be accessed by any other app, including the other apps that are part of this project. The security level setting will dynamically change the behaviour of the vulnerable and malicious apps, and the user does not need to restart them, with some
    exceptions. The challenge settings can be changed at any time while doing a challenge.
    
    \subsection{Identifying the malware and the vulnerable app}
        \label{subsec:identify_challenge_apps}
        
    \begin{wrapfigure}[19]{r}{0.4\textwidth}
        \centering
        \includegraphics[width=0.4\textwidth]{graphics/manifests.PNG}
        \caption{The Manifests page in the Challenge activity.}
        \label{fig:manifests_fragment}
    \end{wrapfigure}
        
    Once the user applies the settings described in subsection \ref{subsec:challenge_settings}, the application starts the Challenge activity, which you can see in figure \ref{fig:manifests_fragment}. The first task of every challenge is to identify the pair of vulnerable and malicious apps for that challenge from amongst all of the apps that are part of the project, except the DVM-ICC app itself. 
    
    The Info page gives a technical description of one of the two major attack categories, Intent Hijacking or Intent Spoofing, followed by a detailed description of the specific cyber attack related to this challenge. After reading and understanding this information, the user needs to go the Manifests page, where they can view for each app the contents of its manifest and all snippets of code that sends intents, as shown in figure \ref{fig:manifests_fragment}. Based on the technical background the user was given in the Info page and in the Home Menu from subsection \ref{subsec:home_menu}, they need to look through the code in the Manifests page and try to identify the vulnerable app and the malware for this challenge. 
    
    The user should be able to detect vulnerable apps by looking for intents being sent implicitly or for exported components, for example. Malwares can be usually identified by the use of intent filters matching intents that are only sent by a vulnerable app, or by sending an explicit intent to another app that is part of this project. The features that identify the vulnerable app and the malware depend on the selected challenge.
    
    The Questions page contains text boxes where the user can type in their answers and be told if they have correctly identified the apps.
    
    \subsection{Performing the cyber-attack}
        \label{subsec:perform_attack}
        
    Now that the user knows the correct pair of apps for this challenge, the next task of the challenge is to use these apps to witness the cyber-attack take place.
    
    \begin{wrapfigure}[20]{r}{0.4\textwidth}
        \centering
        \includegraphics[width=0.4\textwidth]{graphics/log.PNG}
        \caption{Log of the malware for Content Provider URI Hijack after it successfully attacked the vulnerable app.}
        \label{fig:malware_log}
    \end{wrapfigure}
    
    Once you have correctly answered the questions regarding the identity of the malware and vulnerable app, the Instructions page will display detailed instructions for using the apps such that the specific cyber-attack for that challenge will take place. This page can be accessed from the activity shown in figure \ref{fig:manifests_fragment}.
    
    In general, the user is asked to take a look at the two apps to familiarise themselves with them and what they can do. Then, the user should make sure that the security level is set to low. After that, they should use the vulnerable app and malware according to the provided instructions in order to trigger the attack and observe it. 
    
    Following that, the user needs to go the malware and look in its Log, which is accessed from its app bar. If the attack was successful, the malware should have written in its log the flag associated with this specific challenge and security level, along with any relevant data it managed to steal from the vulnerable app. These would be hidden amongst the output of Android Log. 
    
    You can see an example of this in figure \ref{fig:malware_log}, where the messages from the Whatsapp vulnerable app have been stolen by the SMS Messages malware. The user should then copy the flag and submit it to the flag question for the current security level in the Questions page in the DVM-ICC app, to prove they have completed one level of the challenge. 
    
    At this point, the user should clear the log of the malware, to see the effects of the next attack more easily, and then change the security level to the next value. They should repeat the process described in this subsection for every security level of that challenge.
    
    Through performing the task described in this section, the user will witness an example of a cyber attack happening in an authentic scenario, with a real malicious app exploiting an insecure app. 
    
    \subsection{Security Levels Explanation}
        \label{subsec:security_levels_explanation}
        
    \begin{wrapfigure}[22]{r}{0.45\textwidth}
        \centering
        \includegraphics[width=0.45\textwidth]{graphics/security_level_explanation.PNG}
        \caption{Part of the explanation of the security levels for the Broadcast Theft Challenge.}
        \label{fig:security_levels}
    \end{wrapfigure}
        
    While completing the task described in subsection \ref{subsec:perform_attack}, the user can click on the Help button that you can see at the top right in figure \ref{fig:manifests_fragment} to view an explanation of the security levels for the current challenge.
    This view is hidden until the user has identified the pair of apps for the current challenge, as explained in subsection \ref{subsec:identify_challenge_apps}.
    
    For each level, it will explain how it works and how it is more secure than the previous level. Following this explanation is a view of the full manifest of the vulnerable app, as it would be used for that security level. Finally, this view might show code snippets that introduce a vulnerability through the way they send an intent.
        
    You can see part of the Security Levels explanation for the Broadcast Theft challenge in figure \ref{fig:security_levels}, as an example. The top code snippet shows the code used to send a broadcast in the News Aggregator app in the medium security level. Since this code send the broadcast implicitly, it is insecure.
    
    This activity teaches the user how and where the vulnerability is introduced and what steps a programmer can take to secure the vulnerable app, complete with code examples from the app itself. Some solutions fail to completely remove the vulnerability, and this activity teaches the developer why they are inadequate. The explanation of the Impossible level shows how to completely remove the attack surface.
    
    By looking at these explanations while performing the task described in subsection \ref{subsec:perform_attack}, the user will be able to see how the cyber attack of that challenge can still succeed despite the programmer implementing some security defences.
    
    \subsection{Challenge Conclusion}
        \label{subsec:challenge_conclusion}
    
    Once the user has performed the cyber attack and has submitted the correct flag for every security level, they have completed that challenge. As a reward, they can now click on the trophy button on the app bar of the Challenge activity, which you can see in figure \ref{fig:manifests_fragment}.
    
    This will open an activity where the user can read the conclusion to the current challenge. This text will give a summary of the whole challenge, going through how the vulnerability is introduced, how the attacker can create a malware that takes advantage of it, and how to fix the vulnerability. Moreover, the conclusion will comment on the authenticity of the scenario presented in the challenge, covering how or why the vulnerability was introduced in the first place and how likely it is for the scenario might to be encountered in the real world. The purpose of this conclusion is condense all of the information that the user should take away from this challenge.
    
    \subsection{Operation Modes}
        \label{subsec:challenge_modes}
    
    As mentioned at the beginning of section \ref{sec:home_app}, the DVM-ICC application supports three modes. The Beginner mode has already been described in sub\cref{subsec:home_menu,subsec:challenge_settings,subsec:identify_challenge_apps,subsec:perform_attack,subsec:security_levels_explanation,subsec:challenge_conclusion}. It is meant for less experienced users, who need to given detailed explanations in order to understand the ICC vulnerabilities of each challenge.
    
    The Expert Mode works in the same way as the Beginner mode, but it hides a lot of information from the user to increase the difficulty. In the Challenge activity from figure \ref{fig:manifests_fragment} on page \pageref{fig:manifests_fragment}, the Info page will be hidden, and the Security Levels Explanation activity from figure \ref{fig:security_levels} will only contain code snippets, without any of the explanations.
    
    The Make your own Malware mode disables the malware app of the current challenge in order to allow the user to create their own malware and attack the vulnerable app without interference from the included malware. In this mode, the user only has access to the Info page described in subsection \ref{subsec:identify_challenge_apps} and the Security Levels activity described in subsection \ref{subsec:security_levels_explanation}.
    
    \section{Challenges}
        \label{sec:challenges}
        
    Section \ref{sec:home_app} covered the general workflow for using the software of this project, which applies to challenges. In this section, we will focus on the vulnerability explored by each challenge and the pair of malware and vulnerable app that demonstrate it. The first five challenges involve variations of the Intent Hijack attack presented in subsection \ref{subsec:intent_hijack}, while the sixth is a variation of the Intent Spoofing attack explained in subsection \ref{subsec:intent_spoofing}.
    
    For all challenges presented here, the description in the Implementation subsection of each challenge describes the interaction between malware and vulnerable app when the security level is set to Low. The other security levels are presented in the Security Levels subsection of each challenge.
    
    \subsection{Broadcast Theft}
        \label{subsec:broadcast_theft}
        
    When a broadcast is sent, the sender does not receive any indication of what components have received that broadcast. A malicious app can register a broadcast receiver with as many intent filters as possible to listen to many public broadcasts \cite{2010_icc_paper}. The malware can read the data in the broadcasts without the user knowing it, and could therefore be used as spyware. 
    
    Moreover, if an implicit intent is used to make a broadcast meant for an app’s internal use, that broadcast could be sent to any receiver on the device with a matching filter. An attacker could reverse engineer an application, see the intent filter that matches the broadcast, then add an identical filter to a broadcast receiver in their malware.
    
    \subsubsection{Implementation}
        \label{subsubsec:broadcast_theft_implementation}
        
    The vulnerable app for this challenge is News Aggregator, an application that looks for online news articles from various sources and shows them to the user. When its background service has downloaded the news articles, it will send a broadcast with the JSON data, which will be received by a broadcast receiver in the app. This receiver will process the data and update the UI. The broadcast is sent as an implicit intent, and any receiver with an intent filter that includes the action string of the broadcast will get it.
    
    In a real world application, developers might want to use an implicit broadcast for intra-app communication in order
    to make the application more loosely coupled, and thus more modular and flexible.
    Another reason might be that you want other apps made by you to know when the News
    Aggregator has gotten some articles.

    The malware for this challenge is the Call Logger app, which displays the user's call history. In it, the attacker has added a broadcast receiver with an intent filter that is identical to that of News Aggregator's receiver. If Call Logger and News Aggregator are installed on the same device, the broadcast sent by News Aggregator will be received by both its receiver and the malware's receiver. Call Logger can therefore see what news articles a user is looking at and what they might be interested in.
    
    \subsubsection{Security Levels}
        \label{subsubsec:broadcast_theft_security_levels}
    
    When sending a broadcast, the developer can specify the permission that a receiver's app needs to have in order to be able to get the broadcast. This can be used to guard against Broadcast Theft. A developer can declare their own custom permission, as seen in subsection \ref{subsec:custom_permissions}, which must be obtained by receivers to be able to get broadcasts sent by the developer's app. 
    
    If the developer does not set the protection level of the permission, it will default to "normal". This means that the malware will get it automatically at install time if it asks for that permission in its manifest. This is what happens in the Medium security level. In the High security level, the protection level of News Aggregator's custom permissions is set to "dangerous", meaning the malware needs to ask the user to grant that permission. This is quite secure, but a careless user may still grant it. The most secure protection level for the custom permission is "signature", but if the private key of the certificate is not stored securely, it can be stolen and be used to sign the malware, which would be able to get the permission when it is installed and receive the broadcast. This is demonstrated in the Very High security level. The protection levels of permissions were fully explained in subsection \ref{subsec:types_of_permissions}.
    
    Because permissions with a normal or signature permissions are granted at install time, they can not be changed dynamically. Therefore, for the Medium and Very High security levels, the user needs to use the Call Logger 2 and 3 app, respectively. These are identical to Call Logger, except they ask in their manifest for the relevant permission for their security level, and they have a different app icon.
    
    In the Impossible security level, the broadcast is sent as an explicit intent, as the News Aggregator service sets the name of the app whose components can receive the broadcast. This completely removes the vulnerability \cite{2010_icc_paper}. However, it also means that the broadcast can not be received by other apps, which the developer might want, depending on the situation.
    
    \subsection{Broadcast Theft - Denial of Service}
        \label{subsec:broadcast_theft_dos}
        
    The challenge presented in subsection \ref{subsec:broadcast_theft} focused on normal broadcasts. Ordered broadcast are sent to receivers one at a time, as explained in subsection \ref{subsec:receivers}. The use of ordered implicit broadcasts can not only enabled eavesdropping as described in subsection \ref{subsec:broadcast_theft}, but denial of service attacks as well \cite{2010_icc_paper}. A malicious app can register a broadcast receiver with a very high priority to ensure it is the first to receive it. It can extract the data from the broadcast, and then abort the broadcast, ensuring the intended receivers do not get it and thus perform a Denial of Service attack.
    
    \subsubsection{Implementation}
        \label{subsubsec:broadcast_theft_dos_implementation}

    The Call Redirect app is the vulnerable app for this challenge. This app redirects phone calls by automatically adding a customizable country code prefix before the number is dialed. It achieves this by having a broadcast receiver which listens for broadcasts with the action NEW\_OUTGOING\_CALL, sent by the OS whenever a call is being made \cite{intents}. This broadcast is ordered, and it contains the dialed phone number. The Call Redirect reads this number, adds the prefix and then sets the new number to the broadcast.
    
    The malware for this challenge is the CPU Booster app, which pretends to improve the performance of the device, but does not actually do anything. It will also have a receiver that listens for broadcasts with the NEW\_OUTGOING\_CALL action. This receiver will have the maximum possible priority declared in the malware's manifest. Consequently, it will get the broadcast before Call Redirect. It reads the dialed in phone number, then cancels the broadcast. The legitimate app's receiver will not receive it and the phone call will be canceled.
    
    \subsubsection{Security Levels}
        \label{subsubsec:broadcast_theft_dos_security_levels}
        
    This challenge only has Low, High and Impossible security levels.
        
    Since the broadcast is sent by the system and not by the app, we can not protect it with a
    permission. The developer of the Redirect app can set their receiver's priority to the maximum possible value, like the malware does. This way, the receiver of this app will hopefully be called before the malware's receiver, change the number in the broadcast, and then cancel it so the malware does not receive it. The number will still be dialed because when the broadcast was sent by the OS, a final receiver was specified, which gets the broadcast even if it is canceled. This is what the vulnerable app does in the High security Level.

    According to the Android documentation, receivers with the same priority are executed in an arbitrary order \cite{broadcasts_overview}. From our experience, on devices on Android 7.1 or 8, receivers with the same priority are called in alphabetical order of their app package names. For this reason, the package name of the malware is acpu\_booster, to maximise the chance its receiver will get called first, before Call Redirect's receiver can cancel the broadcast. Consequently, the attack can still happen in the High security level. Moreover, other legitimate receivers do not get the ordered broadcast, as Call Redirect canceled it before they had the chance to receive it.

    Therefore, the best way to fix this vulnerability rests with the user. CPU Booster can not receive the broadcast if it does not have permission to access phone calls from the user, and it has no obvious reason it would need it. In the Impossible level, the user should see that the app is suspicious and they should uninstall it or remove its permission. 
    
    \subsection{Broadcast Theft - Man In the Middle}
        \label{subsec:broadcast_theft_mitm}
        
    The vulnerability explored by this challenge is the same as the for the Broadcast Theft - DOS challenge in subsection \ref{subsec:broadcast_theft_dos}. The difference is that the cyber atack has a different outcome. It is a Man in the Middle attack instead of Denial of Service.
    
    \subsubsection{Implementation}
        \label{subsubsec:broadcast_theft_mitm_implementation}

    This challenge's vulnerable app is Call Redirect yet again. The malware for this challenge is the Battery Booster app, which pretends to improve the battery life of the device, and yet again does not actually do anything. It will also have a receiver that listens for broadcasts with the NEW\_OUTGOING\_CALL action, in the same way CPU Booster did. It will get the broadcast before Call Redirect, read the phone number and replace it with a different phone number, then send the permission to the next receiver. The legitimate receiver will receive the broadcast with the injected number and the country code to it. This phone number will then be dialed. Battery booster can be used to stealthily re-direct phone calls to malicious phone numbers.
    
    \subsubsection{Security Levels}
        \label{subsubsec:broadcast_theft_mitm_security_levels}
        
    The Security levels for this challenge are Low, High and Impossible, and they function in the same way as the security levels of Broadcast Theft - DOS, described in subsection \ref{subsec:broadcast_theft_dos}. The only differences are that the malware does not cancel the broadcast and instead changes its data, and that the package name of the malware is abattery\_booster. The latter is to make sure the malware's receiver gets the broadcast first.
    
    \subsection{Activity Intent Hijacking}
        \label{subsec:activity_hijacking}
        
    In this type of attack, a hacker takes advantage of the use implicit intents so that a malicious activity is launched instead of the intended one, by declaring a matching intent filter for their malicious activity in the manifest.
    
    In a sophisticated version of this attack, the attacker can use phishing to steal user credentials \cite{2010_icc_paper}. If an app uses an implicit intent to start its Login activity, a hacker can use an identical intent filter for a fake Login activity in their malware. Since the attacker reverse engineered the vulnerable app, they can make the fake activity look identical to the real one. The implicit intent will match both the fake Log In activity and the legitimate one, and thus prompt the user to choose between them. The attacker can give a confusing name to their malware, or make the icons of the two apps similar, to make the user choose the malware. The unaware user then types in their credentials on the fake Login screen, which are then be sent to the hacker.
    
    \subsubsection{Implementation}
        \label{subsubsec:activity_hijack_implementation}
        
    The vulnerable app for this challenge is Santander, a banking app. After they log in, the user can view details of fictitious bank accounts and can make fictional payments to another account, which will decrease the balance one of the accounts displayed in the app.
    
    Santander has an intent filter for its Login activity so that the user can click on special link that starts the bank app to make a payment. The link contains details of the recipient of the payment, such that these details are already filled in when the app launches and therefore making the payment easier. An implicit intent with this link can be sent from another app. It will match Santander's intent filter and start the Login activity to authenticate the user, before going to the payment screen.

    The attacker decompiled the Santander app to see the code for the Login activity and its intent filter. They made an activity in the Money Manager malware that looks identical to the one in Santander, and added to it an intent filter. Money Manager is an app for tracking and categorising one's income and expenses, but it does not have full functionality.
    
    When you open the app, the user sees a Loading screen, which sends an implicit intent to start the Login activity when it ends. There will be two filters that match with this intent, and therefore the user will be asked to pick between Santander and Money Manager. Both are financial apps, and Money Manager has a similar icon and colour theme, and therefore a careless user might select the malware. Money Manager's Login activity looks identical to Santander's, so the user types in their credentials, which the malware can then send to the attacker. To make it as if nothing suspicious happened, it will say the credentials are incorrect and then start the legitimate Login activity with an explicit intent.
    
    \subsubsection{Security Levels}
        \label{subsubsec:activity_hijack_security_levels}
        
    When sending intents to start an activity or service, you cannot protect them with a permission like you can when you send a broadcast and as we did in subsection \ref{subsubsec:broadcast_theft_security_levels}. Therefore, in the Impossible security level, the solution to fix the vulnerability is for the Santander Loading activity to use an explicit intent to start the Login activity. By doing this, the intent can not be sent to the malware. Explicit intents should be used for communication between components of the same app.

    However, when the user clicks on a link in another app to make a payment, the malware's intent filter will still match that implicit intent and the attack could succeed. For this case, the solution is to use App Links such that when a user clicks on a web link, they are either taken to a web page in a browser or they are taken to an activity in your app \cite{android_app_links}. These are not implemented in the Santander app at the current moment.
    
    \subsection{Content Provider URI Hijacking}
        \label{subsec:provider_uri_hijacking}
        
   Intents can transmit data using a URI and these URIs can point to data stored using a Content Provider. We have also explained that permissions can be set on a per component basis. However, when declaring a Content Provider in the manifest file, the developer can set the \lstinline|android:grantUriPermissions| tag of the \lstinline|<provider>| element to true. This means that the content provider can allow temporary access to data linked to in a URI for a component that does not otherwise have the permission required by the provider. Therefore, external apps can access the content provider, even if the provider is not exported and is thus not accessible by external apps in general.
    
    In order to do this, the intent that transmits the URI link must have the \lstinline|FLAG_GRANT_READ_URI_PERMISSION| or \lstinline|FLAG_GRANT_WRITE_URI_PERMISSION| flags added to it. If this intent is an implicit intent and is intercepted by a malicious component, then that component can read the data and perhaps modify it. This attack can put private information, such as debit card information or a phone number, in the hands of attackers.
    
	\chapter{Reflections and conclusions}
	\label{chap:reflections}
	
	\section{User studies}
	    \label{sec:user_studies}
	    
	To ensure that our project achieves the aims mentioned in subsection \ref{sec:intro_objective}, we ran an informal user study with two of our peers, Constantinos Loizou and Avi Varma. Both are Computer Science students at Swansea University. Constantinos is a third year student while Avi is a Masters student. Both completed the Cryptography and IT Security and Writing Mobile Apps modules, and therefore are familiar with Android development and cyber security. The study consisted in them completing one challenge with minimal guidance from us and then providing feedback. Constantinos completed Broadcast Theft while Avi did Broadcast Theft - MITM.
	
	They both liked using the final product and said that the intended workflow made sense. More importantly, both said that completing the challenge taught them a lot about ICC vulnerabilities and made them aware of a type of attack they did not know about before. Avi said that "It was sufficiently difficult, fun and interactive enough. It achieves more than its purpose and it should be used by more people." Since the education tool succeeds in educationg developers about ICC vulnerabilities in Android, the project can be considered a success.
	
	The problems encountered during the study are relatively minor and are related to the UI or the explanations included. Both said that the technical background would be better organised into various sections rather than a very long text you need to scroll through, and that workflow explanation should have some small improvements. Constantinos complained in particular that the learning curve is very high initially. Another issue that both mentioned was that malware logs are very long and confusing at first.
	
	\section{Reflections on the schedule}
	    \label{sec:schedule_reflection}
	    
	The project schedule submitted as part of the Initial Document on the 2nd of November 2020 is included in Appendix \ref{app:project_schedule}. The project suffered due to delays from the very start. I was on a Year in Industry full time internship until 08.09.2020 and therefore had no little time to research cyber security over the summer. I settled on ICC vulnerabilities only a week before the Initial Document deadline.
	
% Formatting citations properly when they are rendering incorrectly in your PDF can be fiddly,
% espectially when punctuation and international characters are involed. Sometimes multiple re-compilations
% are needed in addition to clearing temporary auxiliary files to see your changes in your document.
% Insert the bibliography using citations contained in the file citations.bib
	\bibintoc%
	\bibliography{citations} 
	
% In the appendix you might include a full code listing for an implemented algorithm that you showed a 
% small chunk of in one of your chapters. If you have extra graphs for ablation style experiments you 
% might enumerate them within the appendix and use \label{name} and \ref{name} to automatically insert 
% the correct section locations when you talk about them in your chapters.
% Within appendix.tex you should use chapters as the top level section dividers.
	\appendix
	\addappheadtotoc
	
\chapter{Project schedule from Initial Document}
\label{app:project_schedule}

    You can see the project schedule we included in the Initial document in \ref{app:project_schedule} on page \pageref{app:project_schedule}. We have distinguished categories of tasks in the schedule through colour coding. Green represents tasks that are directly related to the completion of the project, whether it be research, design, implementation, or testing. Blue bars represent tasks related to the deliverables required by the university besides the Android Image. Grey tasks are periods of time reserved to focus on other courseworks or on exams. Finally, orange tasks are for the completion of secondary project objectives.
    
    \begin{landscape}
    \begin{figure}[H]
    	% [H] means put the figure HERE, directly when you input this code.
    	\centering
    	\includegraphics[width=1\linewidth]{./graphics/schedule.PNG}
    	
    	% Caption is defined with a short and long version. The short 
    	% version is shown in the List of Figures section, and the long 
    	% version is used directly with the figure. 		
    	\caption[Project schedule.]{The Project Schedule we had in the Initial Document for this project.}
    	
    	% For figures, \label should be defined after the caption to ensure 
    	% proper figure numbering.
    	\label{fig:schedule}
    \end{figure}
    \end{landscape}

\chapter{Supplementary Data}
\label{app:supplementary_data}

The results of large ablative studies can often take up a lot of space, even with neat visualization and formatting. Consider putting full results in an appendix chapter and showing excerpts of interesting results in your chapters with detailed analysis. You can use labels and references to refer the reader here for the full data.

	
\end{document}